\chapter{Literature review}

\section{Tissue engineering}
\subsection{Definition and Background of Tissue Engineering}
Tissue engineering can be defined as the “the application of principles and methods of engineering and life sciences toward fundamental understanding of structure–function relationships in normal and pathological mammalian tissues and the development of biological substitutes to restore, maintain, or improve tissue function” \citep{skalak_1988}.

The concept of replacing dysfunctional tissues and organs has been around for millennia, with the first nose transplants occurring as early as 1000 B.C. in ancient India \citep{saltzman_2004}. In recent centuries, major advances in organ transplantation have taken place – even transplants of organs supplied by donors. These strides include tooth transplants, successful skin grafts (which became commonplace in the 1800s), and even the first successful heart transplant which took place in South Africa, 1967 \citep{saltzman_2004}.

Despite the rapid advancements in organ transplantation in recent centuries, all transplants involving organs from donors have a common issue. The problem is that the tissue/organ in need of replacement requires the availability of viable tissues or organs from a donor and in modern times the demand for tissue/organ replacement far exceeds the available supply from donors \citep{saltzman_2004}. This problem can be alleviated using tissue engineering. 

Through tissue engineering, the demand for useable and available donor tissues is mitigated by creating viable tissues (and even organs) through artificial means. This replacement tissue is formed by creating a three-dimensional scaffold to take on the shape of the space of the human (or organism) that needs to be filled with new tissue. These scaffolds can be acellular or seeded with cells prior to formation. The acellular scaffolds rely on the body’s natural ability to regenerate the cells for normal function. Whether seeded or regenerated from the body, the scaffolds (with its contained cells) are intended to replace the functionality of the old or damaged tissue \citep{olson_2011}.

\subsection{Scaffold Requirements}
The main purpose of artificial scaffolding in tissue engineering is to mimic the environment and behaviour of the tissue being replaced. To create artificial scaffolding that recreates these same conditions, the scaffolding needs to be biocompatible, biodegradable, have sufficient architecture and have the same mechanical properties as the native extracellular matrix \citep{obrien_2011}.

The scaffolds must be biocompatible with the cells. This means that the cells should be able to migrate through the scaffold material and be able to function normally within this artificial EM (extracellular matrix). This criterion also means that the engineered tissue should result in minimal immune response from the body once it is integrated (or implanted) with the host tissue \citep{obrien_2011}.

The artificial scaffolds must be biodegradable since the scaffolds are not meant to be permanent implants. The scaffolds are intended to be gradually broken down and slowly replaced with a new extracellular matrix that is formed by the seeded or host’s natural cells. To ensure that the tissue continues to function normally and remain undamaged, the by-products of the scaffold’s degradation must be non-toxic \citep{obrien_2011}.

The architecture of the scaffolds should be sufficient to allow for the cells to exist inside the matrix. This means that it should have an interconnected porous structure, and it should be highly porous. The reason for this, is so that the structure can allow for the diffusion of nutrients to the cells and the diffusion of waste products from the cells through the EM \citep{obrien_2011}.

The scaffold’s mechanical properties should match that of the native extracellular matrix or that of its intended purpose. This involves selecting materials (for scaffolding) that have similar stiffness (Young’s modulus), hardness, etc. The importance of matching mechanical properties to the scaffolding can differ depending on what the engineered tissues intended purpose is. For example, if its function includes a structural role (as would be the case with bones and cartilage), then it is important to ensure it has enough stiffness and strength to not break upon usage. It should also be noted that the artificially formed matrix should have sufficient mechanical properties for it to remain undamaged when being implanted. In addition to this, it is important to ensure that the scaffold’s mechanical properties do not get chosen at the cost of the scaffold porosity as this can result in the engineered tissue having difficulty in allowing vascularization and cell infiltration resulting in conditions where the cells cannot exist over time \citep{obrien_2011}.

\subsection{Scaffold Fabrication Methods}
There are many methods of fabricating scaffolds. These techniques for creating scaffolds differ according to the material used to create the scaffold and depending on whether the seeded cells can survive the scaffolding formation process. The formation process also needs to ensure that the produced scaffolds meet the requirements for being used as engineered tissue (it must be biocompatible, porous, etc.). These varying methods can be split up into conventional and advanced methods \citep{dutta_2017}.

Conventional techniques include particulate-leaching, extrusion, molding, thermally induced gelation, gas foaming, etc. \citep{dutta_2017}. These processes can be used in combination to manufacture scaffoldings that meet the previously identified scaffold criteria and can work together to form more advanced manufacturing techniques.

Advanced methods include 3D printing, electrospinning, emulsion templating, and designed self-assembling peptides:

\begin{itemize}
  \item 3D printing allows for the computer aided design of scaffolding and then the formation of these designs through additive manufacturing methods such as Fused Deposition Modelling, Stereolithography, and Laser Sintering; this style of fabrication is particularly useful for rapid prototyping, developing scaffolds with complex geometries, and having control over the macroscopic properties of the artificial matrix structure \citep{dutta_2017}. 
  \item Electrospinning involves using an extruder with an electric field to produce ultrafine nanofibers to form the extracellular matrix; this method is very useful for creating scaffolds with fibrous nanoporous materials with high precision \citep{dutta_2017}. 
  \item Emulsion templating uses phase separation to form tertiary pores to create scaffolds with a hierarchical pore structure; emulsion templating is useful for creating scaffolds with a porosity gradient \citep{dutta_2017}. 
\end{itemize}

\subsection{Scaffold Materials}
The material used to form the extracellular matrix depends on the application of the tissue that is being engineered. These possible materials include linear aliphatic polyesters and other synthetic polymers, natural macromolecules, inorganic materials, and hydrogels \citep{mapx2004}.

Linear aliphatic polyesters, such as PGA and PLA, are a group of biodegradable plastics that break down due to the hydrolysis of ester bonds. There are also other synthetic materials such as PPF and tyrosine-derived polymers that are used in bone engineering \citep{mapx2004}.

Natural macromolecules, like proteins polysaccharides, are often used in tissue engineering. Examples of fibrous proteins that are used to create tissue is collagen and silkworm silk. Collagens have useful properties and is therefore often used as a major component of EM scaffolding although it has potential issues with pathogen transmission and less controllable biodegradability; silk (from silkworms) is a useful tissue replacement option in certain instances due to its desirable tensile properties but has issues with cytotoxicity and slow degradation. Polysaccharides such as alginate, chitosan, and hyaluronate (these examples are or can be used to create hydrogels) are also useful in creating porous solid-state scaffolds \citep{mapx2004}.

Inorganic materials that fall into the categories of porous bioactive glasses and calcium phosphates are used in the fields of bone and mineralized tissue engineering due to their ability to support cell adhesion, growth, and differentiation \citep{mapx2004}.

Hydrogel polymers (like PEGs, alginates, etc.) are also a desirable option for producing extracellular scaffolds because they can be easily made to form structures of complex shapes/geometries, they can be seeded with cells, have solidification rates that are controllable (to some degree) and can often be implemented through procedures that are minimally invasive \citep{mapx2004}.

\section{Hydrogel}

\subsection{Introduction to Hydrogels}
Hydrogels are three-dimensional polymeric networks that are hydrophilic and therefore capable of containing large quantities of water while keeping their structural integrity \citep{peppas_2000}. Hydrogels have many desirable properties that make it useful in many applications (even outside of tissue engineering).

The network structure of hydrogels is comprised of homopolymer or copolymer chains. These chains attract water, but the overall polymer network is still insoluble due to chemical and/or physical (entanglements, crystallites, etc.) crosslinks. Crosslinks are the tie points where the adjacent or neighboring chains are joined together. These crosslinks allow for the network structures to be solid and have some degree of structural integrity \citep{peppas_2000}.

Hydrogels can be used in many applications. Due to their resemblance to the properties of natural extracellular matrices of tissue (as they are porous, enzymatically degradable, soft in consistency, biocompatible, and have high water content), they are an attractive option for applications within the medical and pharmaceutical sectors. Their biocompatibility (contributed to by its capacity to hold large amounts of water) allows it to make contact lenses, biosensor membranes, artificial heart lining, artificial skin materials, and drug delivery systems \citep{peppas_2000}.

\subsection{Classification of Hydrogels}
Hydrogels can be classed in many ways and there are many kinds of hydrogels with different applications within the fields of medicine or tissue engineering.

Hydrogel can be classified according to its polymer source, physical properties, ionic charge, biodegradability, or its method of crosslinking. With regards to classifying these gels according to their source, they can either be natural, synthetic, or hybrid polymer hydrogels \citep{liwu2020}. 

Natural hydrogels are a form of natural polymer (or biopolymer) as they are derived from organisms. These can be further classed into polysaccharide hydrogels (such as alginate which comes from brown algae), glycosaminoglycan hydrogels, and polypeptide/protein hydrogels (the most common example of these include collagen and gelatin hydrogels) \citep{liwu2020}.

Synthetic hydrogels are artificially created and come in the form of polyacrylamide (PAAm), poly(ethylene glycol) (PEG), and poly(vinyl alcohol) (PVA) hydrogel. PEG derived gels are the most used synthetic hydrogels in medicine due to their hydrophilic nature and excellent biocompatibility \citep{liwu2020}.

Hybrid hydrogels are made from a combination of natural and synthetic materials. The reason for this is that synthetic and natural gels on their own tend to have limited mechanical strength but through the combination of natural and synthetic materials this problem is rectified to allow improved and tunable mechanical properties as well as allow for the addition of other desirable characteristics to the gel product \citep{liwu2020}.

\subsection{Properties of Hydrogels}
Since there are countless types and forms of  hydrogel, in this section, the properties of specific examples of hydrogels, relevant to the context of extrusion-based 3D printing with the intention of engineering tissue, are considered. These examples are GelMA (gelatin methacrylate), PEGDA (poly(ethylene glycol) diacrylate), Alginate, and Matrigel.
\begin{itemize}
  \item Biocompatibility – all four of these hydrogels are highly biocompatible but only some are naturally able to adhere to cells. GelMA and Matrigel support cell adhesion; PEG-based hydrogels and Alginate naturally have poor cell adhesion, and this can be rectified by incorporating cell-adhesive peptides \citep{liwu2020}.
  \item Biodegradability – GelMA, Alginate, and Matrigel are enzymatically degradable. If the necessary enzymes are not available to degrade Alginate,  it can also be ionically degraded while immersed in an aqueous solution with Na+ ions. PEG-based gels are not degradable and need to be chemically modified to become biodegradable \citep{liwu2020}.
  \item Water retention – PEGDA, GelMA, and Alginate are all very soluble (in water and other solvents) and capable of holding large amounts of water \citep{liwu2020}. Matrigel needs to be chilled for it to be more easily soluble \citep{merceron_2015}.
  \item Crosslinking method – PEGDA and GelMA (with a photoinitiator) are made to cure via photo crosslinking when irradiated by UV rays. Alginate is ionically crosslinked by divalent cations like Ca2+ \citep{liwu2020}. Matrigel, however, is thermally crosslinked (which is reversable). It is a liquid at low temperatures (around 4°C), and it forms a solid matrix at higher temperatures of around 37°C \citep{merceron_2015}.
  \item Rheological properties – the rheological properties of hydrogels often depend on its temperature, concentration is a solute, and molecular weight. Before gelation, PEGDA is found to be a Newtonian fluid (to a certain shear flow rate) and its viscosity increases as its molecular weight increases \citep{brikov_2016}. Both GelMA and Alginate, on the other hand, are non-Newtonian fluids that exhibit shear-thinning behaviour, meaning that their viscosity reduces as its shear rate increases \citep{gregory_2022}. The viscosity of GelMA also varies depending on temperature and concentration. Its viscosity decreases as its temperature increases and its concentration decreases. The viscosity of GelMA, therefore, is very low when flowing at temperatures above \(30\,^{\circ}\mathrm{C}\) (significantly less than \(1\,\mathrm{Pa \cdot s}\)), but at cooler temperatures (less than \(30\,^{\circ}\mathrm{C}\)) its viscosity drastically increases \citep{adhikari2021photoinduced}. For example, a solution with the GelMA concentration of 10\% at \(26\,^{\circ}\mathrm{C}\) has a viscosity of roughly \(168\,\mathrm{Pa \cdot s}\), which is a very large increase in viscosity \citep{cellink_gelma_2020}.

\end{itemize}

\section{3D Printing Hydrogel}
As extrusion-based 3D printing has continued to grow as a method of engineering tissues with hydrogel, many companies have developed state-of-the-art hydrogel 3D printers for biomedical applications. The most advanced and notable bioprinters of today include the CELLINK BIO X printers, the RegenHU 3DDiscovery™ Evolution, and the Allevi 3 printer. To get an idea of the state of the art of hydrogel printers, this subsection will discuss the capabilities of the BIO X printer.

The BIO X 3D printer is developed by a company called CELLINK. It is capable of extruding and printing with many kinds of bioinks including GelMA, PEGDA, Alginate, etc. for application in biomedical research and has a print resolution of 1µm \citep{biox_brochure}. 

The BIO X has interchangeable and attachable printheads for different forms of gel extrusions including a pneumatic, syringe pump, thermoplastic and inkjet printheads. The printing bed and printheads are temperature controlled (to account for the wide-ranging rheological properties of different hydrogels). The printers also include UV LED attachments for photo-induced crosslinking and implement a UV ray-based system (with smaller wavelengths than for curing) for sterilizing the printing environment in between prints to ensure biosafety standards are met \citep{biox_brochure}.
