\chapter*{ECSA self-assessment}

\renewcommand{\arraystretch}{1.5}
\begin{center}
\begin{longtable}{l}


\parbox[t]{\linewidth}{
    \textbf{\large GA 1. Problem solving} 
}\\
\parbox[t]{\linewidth}{
    The project will heavily require problem solving through creating solutions to problems like finding a cheap hydrogel alternative for testing, the issue of printing speed due to slow curing rates by investigating and implementing a method of forced and accelerated hardening.
}\\[1ex]
    

\parbox[t]{\linewidth}{
    \textbf{\large GA 2. Application of scientific and engineering knowledge} 
}\\
\parbox[t]{\linewidth}{
    The project will require the application of many fields of science and engineering. It will use knowledge of fluid mechanics to describe the rheological properties of a selected hydrogel which will be used to define the requirements and parameters of the extruder, which will require a mathematical description of the fluid. Designing the extruder and printing gantry will utilize physical knowledge of strengths and materials by implementing a system that can withstand the induced and static loads. This knowledge will required mathematics and numerical methods to model and describe the effects from these fields of science.
}\\[1ex]


\parbox[t]{\linewidth}{
    \textbf{\large GA 3. Engineering Design} 
}\\
\parbox[t]{\linewidth}{
    A detailed design process will be conducted in this project for both designing the machine (mechanical) and embedded system (electrical). This process will include a literature review to gather knowledge required to understand the functionality of the system, a concept generation where three concepts for each subsystem will be developed, followed by an evaluation of the generated concepts that will be used to select the concepts for each system, and a final design will be developed.
}\\[1ex]
    

\parbox[t]{\linewidth}{
    \textbf{\large GA 5. Engineering methods, skills and tools, including Information Technology} 
}\\
\parbox[t]{\linewidth}{
    This attribute will be achieved through the need for mathematically understanding the rheological properties of liquid hydrogel (fluid mechanics), using Inventor as a means of CAD to implement the skill and method of machine design, using an integrated development environment to engage in the design and programming of embedded systems to implement methods of control systems, and using numerical methods for modelling and processing raw data with MATLAB.
}\\[1ex]
    

\parbox[t]{\linewidth}{
    \textbf{\large GA 6. Professional and technical communication} 
}\\
\parbox[t]{\linewidth}{
    The project will include a final report that will clearly and professionally communicate its contents using language intended for its target audience (those trained in the field of mechatronics), that will make use of a literature review (which for this attribute, gives the engineering sufficient knowledge in biomedicine to understand the context/application of the project), graphical aid (CAD drawings, electrical diagrams, and experimental results), design and experimental methodology. 
}\\[1ex]


\parbox[t]{\linewidth}{
    \textbf{\large GA 8. Individual, Team and Multidisciplinary Working} 
}\\
\parbox[t]{\linewidth}{
    The project will demonstrate this attribute by utilizing knowledge from many fields of science and engineering (thermodynamics, fluid mechanics, electronics and embedded systems design, and mechanical design) and requiring careful planning of the project timeline (with a Gantt chart) and project costs (with a budget) to ensure that all the many project deliverables are achieved.
}\\[1ex]
    

\parbox[t]{\linewidth}{
    \textbf{\large GA 9. Independent Learning Ability} 
}\\
\parbox[t]{\linewidth}{
    The project will demonstrate independent learning ability due to the need to apply advanced knowledge of material science (properties of hydrogel), electrical and mechanical (regarding the gantry and extrusion system) design from various academic sources such as textbooks and research papers.
}\\[1ex]


\end{longtable}
\end{center}