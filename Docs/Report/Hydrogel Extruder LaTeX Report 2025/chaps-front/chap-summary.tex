\chapter*{Executive summary}

\renewcommand{\arraystretch}{1.5}
\begin{center}
\begin{longtable}{l}


\parbox[t]{\linewidth}{
    \centering\textbf{\large Title of Project} 
}\\
\parbox[t]{\linewidth}{
    \centering Development of a Hydrogel Extruder
}\\[1ex]
    

\parbox[t]{\linewidth}{
    \centering\textbf{\large Objectives} 
}\\
\parbox[t]{\linewidth}{
    To design and construct a hydrogel extruder, as well as a 3D printer body, with the capability of printing three dimensional shapes (of reasonably complex geometry).
}\\[1ex]


\parbox[t]{\linewidth}{
    \centering\textbf{\large What is current practice and what are its limitations?} 
}\\
\parbox[t]{\linewidth}{
    Its current practice is to make hydrogel structures for tissue engineering and biomedical research. Its limitations include its printing speed due to curing limitations and print resolution. 
}\\[1ex]
    

\parbox[t]{\linewidth}{
    \centering\textbf{\large What is new in this project?} 
}\\
\parbox[t]{\linewidth}{
    A hydrogel printer will be designed and made with the ability to create prints with different kinds of hydrogel and curing methods with a focused calibration of the system to be able to print with a selected test gel to show the design as a working concept. 
}\\[1ex]
    

\parbox[t]{\linewidth}{
    \centering\textbf{\large If the project is successful, how will it make a difference?} 
}\\
\parbox[t]{\linewidth}{
    The project will make a difference by adding new knowledge to the methods of 3D printing (and curing) with hydrogels for medical application as well as provide medical researchers with a new research tool to help create cell cultures and 3D environments to simulate cell growth. 
}\\[1ex]


\parbox[t]{\linewidth}{
    \centering\textbf{\large What are the risks to the project being a success? Why is it expected to be successful?} 
}\\
\parbox[t]{\linewidth}{
    Risks that may prevent the project completion include budgetary and time constraints as well as finding a cheap, easy-to-access hydrogel/surrogate to test and calibrate the printer. 
}\\[1ex]
    

\parbox[t]{\linewidth}{
    \centering\textbf{\large What contributions have/will other students made/make?} 
}\\
\parbox[t]{\linewidth}{
    This is the second iteration of this project, and future students will use this research as a foundation to build more functional and effective printers that are better tools for tissue engineering and biomedical research. 
}\\[1ex]
    

\parbox[t]{\linewidth}{
    \centering\textbf{\large Which aspects of the project will carry on after completion and why?} 
}\\
\parbox[t]{\linewidth}{
    In future applications, the design can be improved, and additional functionality can be added such as implementing additional methods of forced gelation. 
}\\[1ex]
    

\parbox[t]{\linewidth}{
    \centering\textbf{\large What arrangements have been/will be made to expedite continuation?} 
}\\
\parbox[t]{\linewidth}{
    Carefully documenting all details regarding the design (mechanical, electronic, and firmware) as well as conducting and recording an experimental observation to calibrate the printer to a specific gel, and show the printers abilities. 
}\\[1ex]


\end{longtable}
\end{center}